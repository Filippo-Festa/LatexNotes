%&latex
%
\providecommand{\main}{..}
\documentclass[\main/main.tex]{subfiles}

\begin{document}

\section{Langevin equation and FP approach}
\lesson{3}{07/10/2020}
\subsection{The Langevin equation}
There are two theoretical ways to deal with the description of brownian particles: the first one is the Langevin equation.
\begin{eqnarray}
m\frac{dv_i}{dt}=-\underbrace{\Tilde{\gamma}v_i(t)}_{\text{\textbf{I}}}+\underbrace{\Tilde{\eta_i}(t)}_\text{\textbf{II}}
\end{eqnarray}
where \textbf{I} is the term referred to the \textit{macroscopic dissipation}, while \textbf{II} is the term referred to the \textit{stochastic microscopic noise term}.
What we know about the noise $\eta$ is that:
\begin{eqnarray}
\mean{\Tilde{\eta_i}(t)} = 0 
\end{eqnarray}
\begin{eqnarray}
\mean{\Tilde{\eta_i}(t)\Tilde{\eta_j}(t')} &= \Tilde{\Gamma} \delta_{ij}\delta(t-t') 
\end{eqnarray}
which means that the noise is different from zero only if we are dealing with the same component and at the same time. Let's now introduce some rescaled variables:
\begin{eqnarray}
\text{where} \qquad \gamma:=\frac{\Tilde{\gamma}}{m}; \qquad \eta_i:= \frac{\Tilde{\eta_i}}{m}; \qquad \Gamma:=\frac{\Tilde{\Gamma}}{m^2}
\end{eqnarray}
Now we can rewrite the Langevin equation in a more compact way, without the mass of the particle entering the equation:
\begin{eqnarray}
\frac{dv_i}{dt}=-\gamma v_i(t)+\eta_i(t); \quad \mean{\eta_i(t)}=0\,\vee\, \mean{{\eta_i}(t){\eta_j}(t')} &= {\Gamma} \delta_{ij}\delta(t-t') 
\end{eqnarray}
It is important to stress that, when we take the average of the noise, we average \textit{over different realizations of the system} (realizations in the sense of stochasticity).
For a given realization I chose some explicit value for the stochastic variable $\eta$, then I repeat this over a different realization. \\

One can formally integrate the Langevin equation. A formal solution can be written in this way (starting the integral from 0):
\begin{eqnarray}
v_i(t)=\exp(-\gamma t)[v_i(0)+\int_0^t d\tau \exp(\gamma \tau)\eta_i(\tau)]
\end{eqnarray}
Taking the average velocities over the different realizations (i.e. over noises) we get:
\begin{align}
    \mean{v_i(t)}&= v_i(0)\exp(-\gamma t) \\
    \mean{v_i ^2(t)} &= v_i ^2(0)\exp(-2\gamma t) + \frac{\Gamma}{2\gamma}[1-\exp(-2\gamma t)] \\
    &\overset{t\to \infty}{\longrightarrow}\frac{\Gamma}{2\gamma} \label{eq:miao}
\end{align}

Now we can relate this result to the behaviour at the thermodynamic equilibrium: assuming that asymptotically my system reaches thermodynamic equilibrium, we can use the equipartition theorem, which states that: \footnote{Remember that $k_B$ is set to 1 for all the entire course!}
\begin{eqnarray}
\mean{\frac{1}{2}mv_i^2(t)}=T/2
\label{eq:EQPT}
\end{eqnarray}

By comparing (\ref{eq:miao}) and (\ref{eq:EQPT}) we can get a relation between the $\Gamma$ parameter and the temperature:
\begin{eqnarray}
\frac{m\Gamma}{2\gamma} = T \quad \implies \quad \Gamma=\frac{2\gamma}{m}T \quad \implies \quad {\Tilde{\Gamma}= 2\Tilde{\gamma} T}
\end{eqnarray}

This kind of relation is a first example of \textbf{fluctuation-dissipation relation}: the idea is that this relation connects \textit{dissipation} (the friction coefficient $\Tilde{\gamma}$) and \textit{thermal fluctuation at equilibrium} $\Tilde{\Gamma}$ (strenght of the noise). To derive this the assumption of thermal equilibrium was crucial. \\

Let's now compute the MDS (more precisely the MDS part related to the i-th component:

\begin{align}
\mean{(x_i(t)-x_i(0))^2}&=\mean{[\int_0^t v_i(\tau)d\tau]^2}= \\
&=(v_i(0)^2 - \frac{\gamma}{2\gamma})[\frac{1-\exp(-\gamma t)}{\gamma}]^2+\frac{\Gamma}{\gamma^2}t -\\
&-\frac{\Gamma}{\gamma^3}(1-\exp(-\gamma t))
\end{align}

What is relevant here is that we can clearly see the two different regimes:

\begin{itemize}
    \item $\sim v_i^2(0)t^2$ \quad \textcolor{red}{BALLISTIC REGIME\footnote{Taylor expanding the exponential terms we get that distance is scaling linearly with time depending on the initial velocity.}} \quad $t<<t_d=\frac{1}{\gamma}$
    \item $\sim \Gamma/\gamma^2 t:=2Dt$ \quad \textcolor{red}{DIFFUSIVE REGIME\footnote{All the exponential terms go away and the leading term is the one which is linear in time (the second one): asymptotically the MSD is linear in time and so distance scales as the square root of time.}} \quad $t>>t_d$
\end{itemize}
\begin{eqnarray}
D=\frac{\Gamma}{2\gamma^2}=\frac{T}{\gamma}m = \frac{T}{\Tilde{\gamma}}
\end{eqnarray}
where $D=\frac{T}{\Tilde{\gamma}}$ is the \textbf{Einstein relation}. \\

One formally defines the diffusion coefficient $D$ as:
\begin{eqnarray}
\lim_{t \to \infty} \frac{\mean{(\Vec{x}(t) - \Vec{x}(0))^2}}{t}=2dD
\end{eqnarray}

\subsection{Langevin equation with an external conservative force}
Let's now consider an extension of the Langevin equation in the presence of a conservative force. As we know if a force is conservative (not necessarily uniform) $\Vec{F}(\Vec{x})=-\Vec{\nabla}U(\Vec{x})$, where $U$ is a potential energy and essentially now we will introduce the position $x$ (previously the Langevin equation was a differential equation fro the velocities):
\begin{eqnarray}
m\frac{d\Vec{v}}{dt}=m\frac{d^2\Vec{x}}{dt^2}=-\Vec{\nabla}U(\Vec{x})-\Tilde{\gamma}\frac{d\Vec{x}}{dt}+\Tilde{\eta}(t)
\end{eqnarray}

The fluctuations due to the thermal noise is the case of a brownian particle are provided by microscopic collisions of the particle with the constituents of the fluids [thermal fluctuations] and those fluctuations are described using the stochastic (noise term)/force $\eta$.

\subsection{Fokker Plank approach}
In this case the idea of dealing with stochasticity is to deal with probability distribution for the position of the brownian particles at time t $p(\Vec{x},t)$ PDF IN $\Vec{x}$ at time t. Once we know this PDF we can sample it and in this way we can sample different realization of our system. \\

\textit{Notation}: with $W(\Vec{x},t | \Vec{x}',t+\Delta t)$\;\footnote{$W(\text{Starting point} | \text{arrival point})$} we describe a conditional probability: it is the probability of getting to position $\Vec{x}'$ at time $t+\Delta t$ given condition to the fact that at time $t$ my particle was in position $\Vec{x}$. \\

We can use this conditional probability to calculate the \textit{transition rate} $R(\Vec{x}',\Vec{x})$ \; \footnote{Switched order: $R(\text{arrival point|\text{starting point}})$} [$s^{-1}$].
\begin{eqnarray}
R(\Vec{x},\Vec{x}')=\lim_{\Delta t \to 0} \frac{1}{\Delta t} W(\Vec{x},t | \Vec{x}',t+\Delta t)
\end{eqnarray}

\medskip

This rate is a PDF in $\Vec{x}'$, even if it is an object in $\Vec{x}'$ and $\Vec{x}$ because is a PDF considered over the possible arriving points: it originates from a conditional probability so $\Vec{x}$ is somehow fixed but I can consider all possible arrival points (not clear). \\

I can use these rates to write a \textit{master equation}: I want to write an equation for the rate of change of my PDF $p(\Vec{x},t)$ on how the PDF is changing in time and then I can use my transition rates to write the following equation. The point is that I will have two different terms: the \textbf{gain term}, meaning that the rate of change of the probability of finding the particle in $\Vec{x}$ is increasing from processes where the particle is coming from $\Vec{x'}$ to $\Vec{x}$:. This describes a rate transition from $\Vec{x}'$ to $\Vec{x}$: this is the gain term, I integrate over all the starting points. Now $\Vec{x'}$ is the starting point, while $\Vec{x}$ is the arrival point.

On the other hand the \textbf{loss term} were we deal with particles that leave $\Vec{x}$ and go to $\Vec{x}'$. In this case I am integrating all over the possible arrival points.
\begin{eqnarray}
\frac{\partial p(\Vec{x},t)}{\partial t}= \int d^3x' \underbrace{[p(\Vec{x}',t)R(\Vec{x},\Vec{x}')}_\text{{Gain term}}-\underbrace{p(\Vec{x},t)R(\Vec{x}',\Vec{x})}_{\text{Loss term}}]
\end{eqnarray}

The way the FP equation is derived is based on the assumption that the rates are practically non zero only for small values of $\Vec{\chi}:=\Vec{x}'-\Vec{x}$. 

This is reasonable: the probability of growing from $\Vec{x}$ to $\Vec{x}'$ per unit time will be smaller when the distance between the two points involved in the transition becomes larger. \\

Therefore the idea is to Taylor expand the gain term around $\Vec{\chi}=0$ (zero movement) up to the second order:
\begin{eqnarray}
\frac{\partial p(\Vec{x},t)}{\partial t}=-\sum_i \frac{d}{dx_i}(a_i(\Vec{x})p(\Vec{x},t))+\frac{1}{2}\sum_{ij}\frac{\partial^2}{\partial x_i\partial x_j}(b_{ij}(\Vec{x})p(\Vec{x},t))
\end{eqnarray}
The zero order expansion cancels the loss term and one is left with the first and second order expansion terms.
Formally I can relate the quantities $a,b$ to the transition rates such that:
\begin{eqnarray}
a_i(\Vec{x})=\int d^3\chi R(\Vec{x}',\Vec{x})\chi_i \\
b_{ij}(\Vec{x})=\int d^3\chi R(\Vec{x}',\Vec{x}) \chi_i \chi_j
\end{eqnarray}
If we consider as example a brownian particle,
$\Delta x_i=\chi_i$ is the displacement and the factor $a$ is defined as $\Vec{a}=\frac{\mean{\Delta\Vec{x}}}{\Delta
t}$: it's the average displacement per unit of time.

In the general formulation $\vec{a}, b_{ij},R$ could also depend on the time $t$. In the brownian particle case I don't have a dependence on time nor a dependence from $\vec{x}$ and the quantity $\vec{a}$ is uniform and constant (see the following example, $a$ is not always uniform and constant).

Similarly, $b_{ij}=\frac{\mean{\Delta x_i \Delta x_j}}{\Delta t}$ is the average squared displacement per
unit of time. 

How do we relate these quantities to real physical quantities?

If we think about brownian particles subjected to gravity, we have an uniform constant force $\vec{F}_0=-m\vec{g}$, therefore the average displacement $\vec{a}$ is just the \textit{sedimentation speed} (or \textit{drift velocity}):
\begin{eqnarray}
\vec{a}= \vec{v_0}=\frac{\vec{F_0}}{\Tilde{\gamma}} 
\end{eqnarray}
On the other hand we can relate the quantity $b_{ij}$ to the diffusion coefficient: if we assume an homogeneus and isotropic medium then we can say that
\begin{eqnarray}
b_{ij}=2 \delta_{ij}D
\end{eqnarray}

In this simple example (constant uniform force + isotropic and homogeneus solvent) the FP equation is:
\begin{eqnarray}
\frac{\partial p (\vec{x},t)}{\partial t}= -\vec{v_0}\cdot\vec{\nabla} p(\vec{x},t)+D\nabla^2 p(\vec{x},t)
\end{eqnarray}

If we simplify even more, setting to zero the external force, then the FP equation becomes the standard diffusion equation:
\begin{eqnarray}
\frac{\partial p (\vec{x},t)}{\partial t}= D \nabla^2 p(\vec{x},t)
\end{eqnarray}
If we assume the initial condition [in 1d] $p(x,0)=\delta(x-x_0)$ then the solution to the diffusion equation is a gaussian profile whose variance is increasing with time:
\begin{eqnarray}
p(x,t)=\frac{1}{\sqrt{4\pi D t}}\exp(-\frac{(x-x_0)^2}{4D t})
\end{eqnarray}
As a result we can compute the average MSD:
\begin{eqnarray}
\mean{(x-\mean{x})^2}=\mean{x^2}-\mean{x}^2=2Dt
\label{eq:MSD}
\end{eqnarray}

the equation we wrote down for the MSD is exact for any time $t$ but this is just because we chose properly the initial conditions: in general if we chose another non-gaussian initial conditions ($\delta$is gaussian) then what happens is that the PDF becomes a gaussian only asimptotically and therefore also this relation for the MSD (\ref{eq:MSD}) holds only asimptotically.
This is how we deal with the diffusive behaviour of the FP case. \\

Let's now go back to the more general way to write the FP equation, in order to point out an interesting property: let's derive again the fluctuation-dissipation relation in the contest of FP.
\begin{eqnarray}
\frac{\partial p(\vec{x},t)}{\partial t}= - \vec{\nabla}\cdot(\vec{a}(\vec{x},t)p(\vec{x},t))+\frac{1}{2}\sum_{ij}\frac{\partial^2}{\partial x_i \partial x_j} (b_{ij}(\vec{x},t)p(\vec{x},t))
\end{eqnarray}
Now I can write the FP in the form of a conservation law:
\begin{eqnarray}
\frac{\partial p}{\partial t}= -\vec{\nabla}\cdot \vec{J}(\vec{x},t)
\label{eq:continuity}
\end{eqnarray}
In this way we have found the \textit{continuity equation}: the rate of change of my probability density is associated to the flow of the current $\vec{J}$, where 
\begin{eqnarray}
{J}_i(\vec{x},t)=a_i(\vec{x},t)p(\vec{x},t)-\frac{1}{2}\sum_j \frac{\partial}{\partial x_j}(b_{ij}(\vec{x},t)\cdot p(\vec{x},t))
\label{eq:current}
\end{eqnarray}
Considering a conservative (constant but not uniform) external force $\vec{a}(\vec{x},t)= \frac{F(\vec{x})}{\Tilde{\gamma}}=\vec{v_0}(\vec{x})$ and $b_{ij}=2d\delta_{ij}$ for an isotropic/homogeneus medium.
If the force is conservative it amits a potential $U$, so
(\ref{eq:current}) becomes:
\begin{eqnarray}
\vec{J}(\vec{x},t)=\underbrace{\frac{\vec{F}(\vec{x})}{\Tilde{\gamma}}p(\vec{x},t)}_{\text{Drift current}}\qquad -\underbrace{D\vec{\nabla}p}_{\text{Diffusive current (Fick's law)}}
\label{eq:J}
\end{eqnarray}
NB: there is a distincion between \textit{stationary states} and \textit{equilibrium states}.
A \textbf{stationary condition} means that our PDF has no dependence on time: $\frac{\partial p}{\partial t}=0$, which physically is what we expect for $t\to \infty$: whatever happens to our system does not depend on time. The stationary condition for (\ref{eq:continuity}) can be expressed as a vanishing condition on the divergence of the current: $\vec{\nabla}\cdot\vec{J}=0$ \\

A \textbf{(thermodynamic) equilibrium condition} is a stronger one: equilibrium condition implies that no curren s are flowing in the system, so $\vec{J}=0$; as we can see equilibrium implies stationarity. \\

Let's calculate our FP approach at equilibrium (which implies that (\ref{eq:J}) must be zero) and this equation results in an expression for the probability density (whose solution is $p^*$):
\begin{eqnarray}
\frac{\vec{\nabla}p^*}{p^*}=\frac{\vec{F}(\vec{x})}{\Tilde{\gamma D}}= -\frac{\vec{\nabla} U(\vec{x})}{\Tilde{\gamma D}}
\end{eqnarray}
Being $p^*=p^*(\vec{x})$ stationary, it doesn't depend on $t$. 
\begin{eqnarray}
\vec{\nabla}((\ln (p^*)))=-\frac{\vec{\nabla} U}{\Tilde{\gamma} D} \implies p^*(\vec{x})=A\exp(-\frac{U(\vec{x})}{\Tilde{\gamma} D})
\end{eqnarray}
What we have written is nothing but the \textit{Boltzmann distribution} provided that $\Tilde{\gamma}D=k_B T= T$: in this way we are back to Einstein's relation $D=\frac{T}{\Tilde{\gamma}}$.



\end{document}